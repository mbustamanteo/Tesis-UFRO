
{
\Hide
\chapter{Introducción}
}

\begin{titular} 
	\uppercase{
	capítulo 1 \\
	introducción \\
	}
\end{titular}


\section{Antecedentes generales}

Actualmente la democracia es la forma de gobierno preferida por el mundo
occidental, en ésta los gobernantes son elegidos por el pueblo cada cierto tiempo
mediante votaciones. La votación electoral constituye un pilar fundamental en 
la construcción de la sociedad occidental, puesto que es el mecanismo preferido 
para elegir a los gobernantes. Tradicionalmente las elecciones han sido efectuadas 
a lápiz y papel, pero dado el avance tecnológico experimentado durante los últimos 
50 años se ha propuesto introducir elementos electrónicos a las votaciones 
a fin de mejorar el proceso eleccionario.

En los últimos años los gobiernos han ido adoptando sistemas electrónicos y digitales
en sus procesos administrativos, en Chile el caso del Servicio de Impuestos Internos
(http://www.sii.cl) es el emblema de cómo se puede disminuir la burocracia y al mismo
tiempo aumentar la cobertura de servicios utilizando medios electrónicos. 

La pregunta entonces es, si es posible extender el uso de esos sistemas electrónicos
y digitales a las votaciones electorales que definen nuestros gobernantes. La votación 
electrónica o e-voting, se refiere generalmente a diferentes tipos de votación que 
utilizan medios electrónicos para emitir el voto y/o para contar los votos. Estos métodos
aún no están consolidados como alternativa a las votaciones de lápiz y papel, ya que 
existen dudas de que estos sistemas sean capaces de garantizar derechos 
constitucionales y al mismo tiempo evitar fraudes electorales o sabotajes.  

Experiencias reales de votación electrónica hay varias y con diversos resultados. 
A lo largo del tiempo se han utilizado varios sistemas de votación electrónica 
en países como: Brasil, India, Estados Unidos, Estonia, Francia, Venezuela, 
Holanda, Reino Unido, Suiza y Canadá. Estos han sido usados en distintas 
tipos de votaciones: elecciones residenciales, senatoriales, municipales y referéndums. 

Desde el 2000 los avances sobre la construcción de sistemas de votación
electrónica más seguros y confiables se han intensificado de acuerdo a las 
decisiones políticas tomadas por los gobiernos de E.E.U.U. y la Unión Europea, pero 
el consenso en la comunidad es que si bien la votación electrónica permite
mejorar las votaciones tradicionales, aún tiene problemas pendientes a 
solucionar.

\newpage
\section{Descripción del problema}

Dado que internet ahora es usado extensivamente por una larga porción de la población
para operaciones bancarias, comercio, compras de pasajes de transporte, etc. se espera 
que los gobiernos puedan incluir un servicio similar para los trámites que involucren al estado,
como el pago de impuestos o las elecciones de gobernantes. Aparte de las consideraciones
técnicas, un sistema de voto electrónico es ``tan bueno como el pueblo piensa que es'', en 
este sentido los sistemas deben ser capaces de demostrar que son confiables, no sólo a nivel
de seguridad sino también de usabilidad y accesibilidad. 

El problema es entonces, cómo poder garantizar a la población de que los sistemas de votos electrónicos
son \textit{buenos} en el amplio sentido de la palabra. Siendo el software una parte muy importante
de éstos sistemas, debemos ser capaces no sólo de construir software que permita a las naciones
elegir sus gobernantes de forma rápida y eficiente, sino también de poder medir su calidad de forma
objetiva para poder construir sistemas legítimos para el pueblo.

Este trabajo busca abordar el problema de la calidad del software de sistemas de votación electrónica
analizando cuáles son los aspectos que más se han trabajado en este tema, dentro del marco del
estándar de calidad ISO 25010:2011, y a partir de esos resultados producir un conjunto de 
recomendaciones para poder profundizar la noción de calidad de software en futuros modelos y sistemas
de votación electrónica.

\newpage
\section{Objetivos}

Objetivo General: 

\begin{enumerate}

	\item[] Identificar y evaluar los requerimientos no funcionales de modelos y 
	aplicaciones de votación electrónica, usando la metodología de mapeo 
	sistemático.

\end{enumerate}

Objetivos Específicos: 

\begin{enumerate}

	\item Aplicar la metodología de mapeo sistemático acerca de modelos de 
	votación electrónica, que permita identificar los modelos actuales y 
	sus principales características

	\item Evaluar los requerimientos no funcionales aplicables a modelos de 
	votación electrónica 

	\item Proponer un conjunto de recomendaciones para los modelos de 
	votación electrónica, a partir de los antecedentes recopilados y 
	evaluación realizada.

\end{enumerate}
