
{
\Hide
\chapter{Conclusiones}
}

\begin{titular} 
	\uppercase{
	capítulo 5 \\
	Conclusiones \\
	}
\end{titular}

{
	\fontsize{22pt}{26.4pt}%
	\selectfont%
	Conclusiones
}

Este trabajo ha presentado un mapeo sistemático que resume la presencia
de requerimientos no funcionales en propuestas de modelos y sistemas
de votación electrónica. De acuerdo a los objetivos planteados y el 
desarrollo realizado en base a lo anterior, es posible concluir: 

El tema de la votación electrónica puede ser visto como un problema tecnológico, donde
se incorporan temas de seguridad informática, eficiencia, confiabilidad y recursos humanos. Sin embargo,
al ser la votación un tema fundamental de la política, las cosas no son tan simples. La decisión
de adoptar sistemas electrónicos de votación en elecciones nacionales tiene fuertes opositores, quienes
creen que la actual tecnología no está a la altura para satisfacer la integridad y transparencia de éstas, pero a
la vez, distintos actores abogan por replicar el éxito de los sistemas electrónicos en el sistema financiero
para poder aumentar la participación ciudadana en los procesos políticos. La esencia de la
democracia es que todos acepten los resultados de las elecciones, incluso cuando se pierde, por tanto
es vital resguardar la integridad del proceso. ¿Cómo integrar los avances 
tecnológicos y a la vez mantener la integridad de las elecciones? En la literatura encontramos
propuestas que combinan los dos deseos pero estamos lejos de hacerlo realidad.
	
En los últimos 10 años ha existido un esfuerzo constante en la academia por mejorar 
los sistemas de votación electrónica, principalmente planteando modelos que sean más seguros
y eficientes. De un total de 60 propuestas seleccionadas hemos podido desprender algunas 
conclusiones de acuerdo a los resultados, ya que al analizar los modelos de votación electrónica 
bajo el marco de características de calidad del estándar ISO/IEC 25010, vemos una
predominante presencia de la seguridad, seguida por eficiencia de rendimiento. 
	
El principal resultado del trabajo es la predominancia de seguridad como requerimiento no 
funcional de los modelos, el cual es consistente con la literatura al señalar que es crítico 
para el éxito en la adopción de estos sistemas que garanticen a los votantes el resguardo 
de sus derechos constitucionales.
	
Acerca de la metodología empleada, el mapeo sistemático es novedoso por cuanto a que 
no se han encontrado trabajos que lo apliquen al tema de votación electrónica y requerimientos no funcionales,
ademas de ser fructífero puesto que 
permite abstraerse de varios sesgos que existen en la votación electrónica, 
considerando las diferencias que existen tanto en marco regulatorio como en necesidades prácticas
de los sistemas a implementar en las distintas naciones. Esta metodología permite ademas
reflejar con mayor certeza los avances generales que ocurren en la academia, por tanto tiene el potencial
de poder explicar los problemas que aún no están resueltos. Consideramos que los resultados
de este trabajo ilustran en parte los beneficios.

Los resultados obtenidos muestran que los aspectos de seguridad de la votación electrónica
no están suficientemente desarrollados para poder satisfacer todos los requisitos impuestos, tanto en 
modelos teóricos como en implementaciones de sistemas. El mayor potencial de la votación electrónica
es la posibilidad de conducir elecciones transnacionales, o elecciones locales con una frecuencia
mucho mayor, no ha sido tomado en cuenta.		

Dado que este mapeo sistemático ha sido realizado considerando sólo
los estudios de investigación publicados en librerías digitales, revistas científicas y actas 
de congreso, sería beneficioso si trabajos futuros abordaran otros tipos de documentos como
\textit{technical reports} y libros.




 


