

{
\fontsize{16pt}{19.2pt}%
\selectfont%
\center%
\begingroup%

%\uppercase{
%dedicatoria 
%}
\endgroup
\endcenter
}

{
\topskip0pt
\vspace*{\fill}
\hfill \textit{A mi abuela Joaquina}
\vspace*{\fill}
}

\clearpage

\begin{comment}
{
\fontsize{16pt}{19.2pt}%
\selectfont%
\center%
\begingroup%

\uppercase{
	agradecimientos 
}
	

\endgroup
\endcenter
}


\clearpage

\end{comment}

{
\fontsize{16pt}{19.2pt}%
\selectfont%
\center%
\begingroup%

\uppercase{
	resumen 
}	

\endgroup
\endcenter
}

\setlength{\parindent}{1cm}
\setlength{\parskip}{5pt}

La votación electrónica se entiende como el uso de medios electrónicos en votaciones
generales donde el pueblo elige a sus gobernantes. Esta idea no es nueva y hace muchos años
que la academia busca reproducir el éxito de los sistemas electrónicos en la banca y el sistema
financiero. 

Este trabajo tiene como objetivo general identificar y evaluar los requerimientos no funcionales de 
modelos y aplicaciones de votación electrónica, usando la metodología de mapeo sistemático. Para lograr 
este objetivo, utilizamos la metodología del mapeo sistemático de la literatura para encontrar
las publicaciones relevantes y el estándar ISO/IEC 25010:2011 como modelo de calidad 
para evaluar los modelos de votación electrónica.

Buscamos en 3 grandes bibliotecas digitales,ACM Digital Library, IEEEXplore y ScienceDirect,
usando las palabras clave ``Electronic voting'', ``E-vote'',``Model'',``Scheme'',``Proposal'',``Protocol''. 
Inicialmente la búsqueda arrojó 240 publicaciones, pero después de discriminar según los 
criterios de inclusión y exclusión se obtuvieron un total de 101 publicaciones con modelos de 
votación electrónica y 60 de éstas tenían relación con requerimientos no funcionales, siendo éstas 
las cuales formaron parte del mapa final. 

Los requerimientos no funcionales elegidos pertenecían a las características de Security, Performance efficiency,
Reliability y Operability. Las subcaracterísticas que se mapearon fueron: Non-repudiation, Fault tolerance,
Resource behavior, Ease of Use, Confidentiality, Integrity, Authenticity y Security Compliance. Éstas fueron elegidas
a partir de varias publicaciones que sugerían que este conjunto era crítico para el éxito en la adopción
de los sistemas de votación electrónica.

Los resultados arrojaron que un 49\% corresponde a la característica de Security, que es consistente con
la literatura, seguido por un 23\% de publicaciones que abordaban Performance Efficiency, revelando que
el principal problema al día de hoy es poder construir un modelo de votación electrónica que sea seguro y
que pueda ser implementada a gran escala dedicándole una cantidad de recursos razonables.

Las recomendaciones derivadas del mapeo son 3: Se presentó una jerarquización de las características
de calidad del estándar ISO/IEC 25010 con el objetivo de orientar a los futuros modelos 
de votación electrónica sobre los aspectos que se deben resolver. Segundo, implementar sistemas de votación basados en software 
de código abierto, puesto que ayuda a la comprensión completa del sistema y permite realizar auditorías de 
seguridad y aplicar las correcciones necesarias, cosa que en los sistemas de código cerrado no es posible 
verificar. Por último, abordar la propuesta de sistemas de votación electrónica de
forma holística, ya que dada la complejidad de éstos sistemas algunos problemas que necesitan ser abordados
aparecen en otros niveles que no son cubiertos por las propuestas analizadas, para esto se presentó
un diagrama de modelado basado en objetivos usando la notación i*.  

La aplicación de la metodología de mapeo sistemático de la literatura resultó ser fructífera en cuanto
es posible visualizar los avances en los modelos de sistemas de votación electrónica, abstrayéndose de 
las distintas diferencias que aparecen entre las investigaciones al considerar los resultados de este trabajo.






